% Options for packages loaded elsewhere
\PassOptionsToPackage{unicode}{hyperref}
\PassOptionsToPackage{hyphens}{url}
%
\documentclass[
]{article}
\usepackage{amsmath,amssymb}
\usepackage{iftex}
\ifPDFTeX
  \usepackage[T1]{fontenc}
  \usepackage[utf8]{inputenc}
  \usepackage{textcomp} % provide euro and other symbols
\else % if luatex or xetex
  \usepackage{unicode-math} % this also loads fontspec
  \defaultfontfeatures{Scale=MatchLowercase}
  \defaultfontfeatures[\rmfamily]{Ligatures=TeX,Scale=1}
\fi
\usepackage{lmodern}
\ifPDFTeX\else
  % xetex/luatex font selection
\fi
% Use upquote if available, for straight quotes in verbatim environments
\IfFileExists{upquote.sty}{\usepackage{upquote}}{}
\IfFileExists{microtype.sty}{% use microtype if available
  \usepackage[]{microtype}
  \UseMicrotypeSet[protrusion]{basicmath} % disable protrusion for tt fonts
}{}
\makeatletter
\@ifundefined{KOMAClassName}{% if non-KOMA class
  \IfFileExists{parskip.sty}{%
    \usepackage{parskip}
  }{% else
    \setlength{\parindent}{0pt}
    \setlength{\parskip}{6pt plus 2pt minus 1pt}}
}{% if KOMA class
  \KOMAoptions{parskip=half}}
\makeatother
\usepackage{xcolor}
\usepackage[margin=1in]{geometry}
\usepackage{graphicx}
\makeatletter
\def\maxwidth{\ifdim\Gin@nat@width>\linewidth\linewidth\else\Gin@nat@width\fi}
\def\maxheight{\ifdim\Gin@nat@height>\textheight\textheight\else\Gin@nat@height\fi}
\makeatother
% Scale images if necessary, so that they will not overflow the page
% margins by default, and it is still possible to overwrite the defaults
% using explicit options in \includegraphics[width, height, ...]{}
\setkeys{Gin}{width=\maxwidth,height=\maxheight,keepaspectratio}
% Set default figure placement to htbp
\makeatletter
\def\fps@figure{htbp}
\makeatother
\setlength{\emergencystretch}{3em} % prevent overfull lines
\providecommand{\tightlist}{%
  \setlength{\itemsep}{0pt}\setlength{\parskip}{0pt}}
\setcounter{secnumdepth}{-\maxdimen} % remove section numbering
\ifLuaTeX
  \usepackage{selnolig}  % disable illegal ligatures
\fi
\usepackage{bookmark}
\IfFileExists{xurl.sty}{\usepackage{xurl}}{} % add URL line breaks if available
\urlstyle{same}
\hypersetup{
  pdftitle={CAFO's Imputing Maps},
  hidelinks,
  pdfcreator={LaTeX via pandoc}}

\title{CAFO's Imputing Maps}
\author{}
\date{\vspace{-2.5em}2024-05-13}

\begin{document}
\maketitle

\subsection{Procedure}\label{procedure}

In order to build the model to impute the missing values we began with
reading in the 2022 USDA: Census of Agriculture
(``\url{https://www.nass.usda.gov/Publications/AgCensus/2022/Full_Report/Volume_1,_Chapter_2_County_Level/}'').
Using python's request library, I looped through the agricultural census
data for each state, scraping the relevant animal and farm counts for
each county and state total.

From there, we conducted preliminary EDA, using R's ggplot2 library to
visualize animal and farm counts for each animal with both state and
county maps. Noticing that some counties and states had values of D,
seemingly representing missing animal counties for certain counties and
states, we also created maps visualizing the missing data along side the
animal count data.

Seeing the large number of counties and some states that had D value for
certain animals, we decided to explore imputing the missing values using
a number of features, such as surrounding counties' area, surrounding
counties' animal count, surrounding counties' population, state animal
count, county area, county farm count, population of county to build out
a linear model that would both impute these values and give a sense for
each of these different covariates and how they impact the animal counts
within the county.

To find the data on surrounding counties, the County Adjacency File
(``\url{https://www.census.gov/geographies/reference-files/time-series/geo/county-adjacency.html}'')
was used which included a list of every counties and those that bordered
it. From here, we summed the population, area and animal counts for
every county's surrounding counties and added it to our initial Census
of Agricultural dataset.

The two linear models we looked into were the Negative Binomial Linear
Regression model and the Poisson Linear Regression Model. Both of these
models excel for count data as they are limited to not be zero. The
Poisson Linear Regression model makes the assumption that the data's
mean roughly equal the data's variance, with the ability to handle the
large number of counties that have zero animals. On the other handle,
Negative Binomial Linear Regression doesn't require the assumption of
the data's mean equaling its variance, as it has an over dispersion term
to control the extent of the variation.

\subsection{Map Plots}\label{map-plots}

\subsection{Negative Binomial Code}\label{negative-binomial-code}

\subsubsection{Sheep Negative Binomial}\label{sheep-negative-binomial}

\subsubsection{Hog Negative Binomial}\label{hog-negative-binomial}

\subsubsection{Broiler Negative
Binomial}\label{broiler-negative-binomial}

\subsubsection{Layer Negative Binomial}\label{layer-negative-binomial}

\subsection{Zero Inflated Poisson
Model}\label{zero-inflated-poisson-model}

\end{document}
